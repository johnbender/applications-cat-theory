\documentclass[11hpt]{article}
\usepackage{amsmath}
\usepackage{mathtools}
\usepackage[all]{xy}
\usepackage{tikz}
\usetikzlibrary{positioning,shapes,shadows,arrows}

\begin{document}

\tikzstyle{abstract}=[rectangle, draw=black, rounded corners, fill=orange,
        text centered, anchor=north, text=white, text width=3cm, ]

\tikzstyle{legend-target}=[circle, draw=black, fill=orange,
        text centered, anchor=north, text=white, text width=0.25cm ]

\tikzstyle{fncall}=[->, -to, double]
\tikzstyle{protolookup}=[->, -to, thick]

\section{functor formulae}

\begin{equation} \label{eq:functor-arrows}
  \xymatrix{ F(A) \ar@/^/[r]^{F(f \circ g)} \ar@/_/[r]_{F(f) \circ F(g)} & F(C) }
\end{equation}
\begin{equation} \label{eq:functor-equate}
  F(f) \circ F(g) = F(f \circ g)
\end{equation}
\begin{equation} \label{eq:functor-id}
  F(id_{a}) = id_{F(a)}
\end{equation}
\begin{equation} \label{eq:functor-html-jqry}
ob(Html) \to ob(Jqry)
\end{equation}

\begin{equation} \label{eq:functor-html-jqry}
hom(Html) \to hom(Jqry)
\end{equation}

\begin{equation} \label{eq:functor-html-jqry}
Html \to Jqry
\end{equation}

\section{category laws, composition}
\begin{equation}
f: B \to C
\end{equation}
\begin{equation}
g: A \to B
\end{equation}
\begin{equation}
h = f \circ g
\end{equation}
\begin{equation}
f,h,g \in hom(\mathcal{C})
\end{equation}

\section{category laws, identity}
\begin{equation}
id_{a}: A \to A
\end{equation}
\begin{equation}
id_{b}: B \to B
\end{equation}
\begin{equation}
g: A \to B
\end{equation}
\begin{equation}
g \circ id_{a} = id_{b} \circ g
\end{equation}

\section{category formulae}
\begin{equation}
ob(\mathcal{C}),\ hom(\mathcal{C})\ |\ map(\mathcal{C})
\end{equation}
\begin{equation}
ob_{\mathcal{C}},\ hom_{\mathcal{C}}\ |\ map_{\mathcal{C}}
\end{equation}


\section{prototype configuration formulae}

\begin{equation}
 f \in hom(Jqry)
\end{equation}
\begin{equation} \label{eq:jquery-default}
 jQuery.fn.init \xmapsto{\ f_{\sigma}\ } jQuery.fn
\end{equation}
\begin{equation} \label{eq:jquery-with-proxy}
 jQuery.fn.init \xmapsto{\ f_{\sigma}\ } Proxy \xleftrightharpoons{\ f\ } jQuery.fn
\end{equation}

\section{diagrams}

\begin{tikzpicture}

  \node[legend-target] (snd-object) at (-1, 2) {
    \textbf{B}
  };

  \node[legend-target] (fst-object) at (-1, 0) {
    \textbf{A}
  };

  \draw[fncall] (fst-object) -| (snd-object);

  \node[legend-target] (proto) at (0, 2) {
    \textbf{B}
  };

  \node[legend-target] (instance) at (0, 0) {
    \textbf{A}
  };

  \draw[protolookup] (instance.north) -| (proto.south);

  \node[abstract] (jQueryFn) at (3, 2) {
    \textbf{jQuery.fn}
  };

  \node[abstract] (jQueryFnInit) at (3, 0) {
      \textbf{jQuery.fn.init}
  };

  \draw[protolookup] (jQueryFnInit.north) -| (jQueryFn.south);

  \node[abstract] (jQueryFn) at (7, 2) {
    \textbf{jQuery.fn}
  };

  \node[abstract] (LazyWarning) at (7, 1) {
    \textbf{LazyWarning}
  };

  \node[abstract] (jQueryFnInit) at (7, 0) {
      \textbf{jQuery.fn.init}
  };

  \draw[fncall] (LazyWarning.north) -| (jQueryFn.south);
  \draw[protolookup] (jQueryFnInit.north) -| (LazyWarning.south);
\end{tikzpicture}

\end{document}

