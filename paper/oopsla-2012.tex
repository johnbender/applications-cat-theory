%-----------------------------------------------------------------------------
%
%               Template for sigplanconf LaTeX Class
%
% Name:         sigplanconf-template.tex
%
% Purpose:      A template for sigplanconf.cls, which is a LaTeX 2e class
%               file for SIGPLAN conference proceedings.
%
% Guide:        Refer to "Author's Guide to the ACM SIGPLAN Class,"
%               sigplanconf-guide.pdf
%
% Author:       Paul C. Anagnostopoulos
%               Windfall Software
%               978 371-2316
%               paul@windfall.com
%
% Created:      15 February 2005
%
%-----------------------------------------------------------------------------


\documentclass[preprint]{sigplanconf}
\nocaptionrule

% The following \documentclass options may be useful:
%
% 10pt          To set in 10-point type instead of 9-point.
% 11pt          To set in 11-point type instead of 9-point.
% authoryear    To obtain author/year citation style instead of numeric.

\usepackage{amsmath}
\usepackage{mathtools}
\usepackage[all]{xy}
\usepackage{url}

\begin{document}

\conferenceinfo{OOPSLA 2012}{October 19--26, Tucson, AZ}
\copyrightyear{2012}
\copyrightdata{[to be supplied]}

% These are ignored unless
% 'preprint' option specified.
\preprintfooter{Identifying jQuery Performance Optimizations}

\title{Automatically Identifying Performance Optimizations for jQuery Based JavaScript Applications}

\authorinfo{John Bender}
           {unaffiliated}
           {john.m.bender@gmail.com}

\maketitle

\begin{abstract}
The popular \cite{bib:usage} jQuery JavaScript library makes the manipulation of HTML documents easy and intuitive, but an unfortunate side effect of its fluent interface is that an unwary programmer can easily introduce unnecessary performance overhead. And while the JavaScript execution in desktop browsers is become fast enough to hide much of the problem, the growing complexity of HTML documents and the ubiquity of web enabled mobile devices have returned performance to a place of importance in the JavaScript development community. We address this issue using category theory to identify a set of JavaScript functions and jQuery methods that can be optimized using function composition to effect loop fusion. In addition, we propose a set of guidelines and derive a library from those guidelines to help developers quickly identify opportunities to make these optimizations. As a result the developer leveraging the the technique and library should be able to spend less time focusing on performance and more time on adding features and value to his or her applications.
\end{abstract}

\category{D.2.7}{Distribution, Maintenance, and Enhancement}{Enhancement}
\category{D.2.2}{Design Tools and Techniques}{Software libraries}

\terms
Performance, Design

\keywords
JavaScript, Category Theory, Loop Fusion, Optimization

\section{Introduction}

JavaScript that leverages the jQuery library can often be identified by its fluency. That is, users are encouraged to make alterations to jQuery objects by ``chaining'' methods and jQuery extension authors are counseled to always return the altered jQuery object to facilitate this form of serial method invocation \cite{bib:chaining}.

\begin{figure}[!ht]
\small
\begin{verbatim}
jQuery( "div" ).hide().addClass( "foo" ).show();
\end{verbatim}
\nocaptionrule \caption{Sample jQuery method chain}
\label{fig:jquery-sample}
\end{figure}

In Figure \ref{fig:jquery-sample}, all \textbf{HTMLDivElement}s in the document are retrieved using the \verb|"div"| CSS selector and used to instantiate a jQuery ``object-set''. They are then hidden, altered to add an additional CSS class, and shown again. Each method invocation, \verb|hide|, \verb|addClass|, and \verb|show| alters \textit{all} the elements in the jQuery object-set and then provides them for the next method to do the same. More concretely, if \begin{math}n\end{math} methods of this form are invoked in sequence it will require \begin{math}n\end{math} full iterations over the object-set.

Given that each of the methods that behave in this fashion iterates over the entirety of the jQuery object-set it's easy to see where a long sequence of them presents an opportunity for optimization by reducing the number of total iterations. Here, we hope to aid developers in reducing the number of full iterations undertaken with long method chains while otherwise maintaining the appealing fluent interface:

\begin{enumerate}
\item We define two categories \textbf{Html} and \textbf{Jqry} along with a Functor that maps from \textbf{Html} to \textbf{Jqry}. In doing so we will identify one set of JavaScript functions, the morphisms of \textbf{Html}, and their jQuery method counterparts, the morphisms of \textbf{Jqry}, that can be composed while remaining confident in the effects of their execution.
\item We define a simple set of guidelines for jQuery method and plug-in developers to follow that will allow end users to manually optimize their method chains where appropriate.
\item We construct a simple and unobtrusive utility for JavaScript developers using jQuery that will alert them to possible opportunities for optimization when chaining methods that adhere to the proposed guidelines.
\end{enumerate}

\section{jQuery Object Method Definitions}

To facilitate understanding of the machinery that underpins jQuery methods, it's useful to understand the two general forms that jQuery methods take when operating on the entire set of elements. The first iterates over the jQuery object-set and relies on side effects to communicate alterations of each constituent element back into the the parent jQuery object (See Appendix A for more). The second maps JavaScript functions over the \textbf{HTMLElement}s belonging to the jQuery object-set. Hereafter referred to as the map-form, this is the form we will address during the couse of the paper. Figure \ref{fig:map-form} illustrates a simple example that uses the jQuery method \verb|map|.

\begin{figure}[!ht]
\small
\begin{verbatim}
jQuery.fn.mapForm = function(){
  return this.map(function( index, htmlElement ) {
    // some alteration to the html element
    return htmlElement;
  });
};
\end{verbatim}
\nocaptionrule \caption{General map-form}
\label{fig:map-form}
\end{figure}

As with all methods defined on \verb|jQuery.fn|, \verb|this| represents an instance of the jQuery object-set. The invocation of \verb|map| on \verb|this| takes a function argument, here a closure, and passes in the index and HTMLElement of each element in the jQuery object-set. Upon successful iteration over the whole set of elements the result is returned for any subsequent method calls in a chain.

\begin{figure}[!ht]
\small
\begin{verbatim}
jQuery.fn.naiveAddClass = function( cls ){
  return this.map(function( i, e ) {
    var old = e.getAttribute( "class" );
    e.setAttribute( "class", old + " " + cls );
    return e;
  });
};
\end{verbatim}
\nocaptionrule \caption{A naive reimplementation of \$.fn.addClass}
\label{fig:naive-add-class}
\end{figure}

Figure \ref{fig:naive-add-class} shows a naive reimplementation of the \verb|addClass| method using the map-form. Here \verb|cls| is the name of a new CSS class to be added to the element and \verb|e| is each HTMLElement in the jQuery object-set. As defined it can be used as a drop in replacement for the predefined version of \verb|addClass| provided by jQuery itself in \ref{fig:jquery-sample}.

\section{Html, Jqry, and a Functor}

To define \textbf{Html} and \textbf{Jqry} we must define the classes \begin{math}ob(\mathbf{Html})\end{math} of objects and \begin{math}hom(\mathbf{Html})\end{math} of morphisms for each. Then for both we must show that an identity morphism exists, composition is possible for the morphisms, that composition is associative, and finally that the set of morphisms is closed under composition \cite[p. ~1]{bib:category-definition}.

To start we define \begin{math}ob(\mathbf{Html})\end{math} as the set of the \textbf{HTMLElement} type and all its subclasses provided by a standards compliant browser's JavaScript API \cite{bib:htmlelement, bib:all-htmlelements}. We also define \begin{math}hom(\mathbf{Html})\end{math} as JavaScript functions from and to objects in \begin{math}ob(\mathbf{Html})\end{math}.

Next, we define the identity function for \textbf{Html} in the manner you would expect:

\begin{figure}[!ht]
\small
\begin{verbatim}
function id( a ) {
  return a;
}
\end{verbatim}
\nocaptionrule \caption{The identity morphism in Html}
\end{figure}

The composition operation is nearly as simple, returning a new closure that, when invoked, will process the function execution in the expected order (See Appendix B for an example):

\begin{figure}[!ht]
\small
\begin{verbatim}
function compose( f, g ){
  return function( a ) {
    return f(g(a));
  };
}
\end{verbatim}
\nocaptionrule \caption{The compose operation for Html}
\end{figure}

To show associativity it suffices that, given different associations, the reduction remains the same, see Figure \ref{fig:html-associativity}.

\begin{figure}[!ht]
\begin{displaymath}
\begin{aligned}
(f\ \circ\ g \circ\ h)(x)\ &= f\ \circ\ g(h(x)) &= f(g(h(x)))\\
cmps(f,cmps(g,h))(x)\ &= cmps(f, g(h(x))) &= f(g(h(x)))\\
\end{aligned}
\end{displaymath}

\begin{displaymath}
\begin{aligned}
cmps(f,cmps(g,h))(x)\ &= cmps(cmps(f,g),h)(x) \\
cmps(f,g(h(x)))\ &= cmps(f(g),h(x)) \\
f(g(h(x)))\ &= f(g(h(x)))\\
\end{aligned}
\end{displaymath}
\nocaptionrule \caption{Reduction of composition}
\label{fig:html-associativity}
\end{figure}

Finally we know that the morphisms in \textbf{Html} are closed under composition because the source and target objects for each morphism both exist in \begin{math}ob(\mathbf{Html})\end{math} as \textbf{HTMLElement} or one of its subclasses.

To define \textbf{Jqry} we must demonstrate the same properties. The class \begin{math}ob(\mathbf{Jqry})\end{math} has a single object, the type of the jQuery constructor, \textbf{jQuery}. The class \begin{math}hom(\mathbf{Jqry})\end{math} is all morphisms from \textbf{jQuery} to \textbf{jQuery}. As before we first define the identity function and then the composition operation in Figure \ref{fig:jquery-id-compose}.

\begin{figure}[!ht]
\small
\begin{verbatim}
jQuery.fn.id = function() {
  return this;
}

jQuery.compose = function( f, g ){
  return function() {
    return f.call(g.call(this));
  };
}
\end{verbatim}
\nocaptionrule \caption{Identity and the compose operation for Jqry}
\label{fig:jquery-id-compose}
\end{figure}

These are distinct from their \textbf{Html} counterparts in that they ignore any parameters and rely on \verb|this| as an implicit parameter. This stems from the use of the morphisms in \textbf{Jqry} with the JavaScript dot notation which dictates the value of \verb|this| and will become important later when discussing the Functor from \textbf{Html} to \textbf{Jqry}.


\begin{figure}[!ht]
\small
\begin{verbatim}
var divs = $( "div" );

assert(divs.id() == divs);

jQuery.fn.hideIdentity =
  jQuery.compose(jQuery.fn.id, jQuery.fn.hide);

assert(divs.hideIdentity() == divs.hide().id());
\end{verbatim}
\nocaptionrule \caption{Invocations of identity and compose Jqry}
\label{fig:sample-jqry-compose}
\end{figure}

Again, we take associativity to be evident by reduction and we know that \textbf{Jqry} morphisms are closed under composition because the source and target object is the same.

Next we define a Functor from \textbf{Html} to \textbf{Jqry} which will serve the dual purpose of exposing the proposed performance optimization and adding rigor to the definition of the morphisms in \textbf{Jqry} that can safely be optimized. The definition consists of two parts, one that operates on the objects of \textbf{Html}, Figure \ref{fig:functor-for-object}, and one that operates on the morphisms, Figure \ref{fig:functor-for-morphism}.

\begin{figure}[!ht]
\small
\begin{verbatim}
jQuery( document.getElementById("foo") );
\end{verbatim}
\nocaptionrule \caption{Functor definition}
\label{fig:functor-for-object}
\end{figure}

\begin{figure}[!ht]
\small
\begin{verbatim}
function functor( morphism ){
  jQuery.map(this, morphism);
}
\end{verbatim}
\nocaptionrule \caption{Functor definition}
\label{fig:functor-for-morphism}
\end{figure}

The Functor must also preserve identity and composition \cite[p. ~36]{bib:category-definition}. The preservation of identity and composition is made obvious through simple illustration (Figure \ref{fig:id-and-compose}).

\begin{figure}[!ht]
\small
\begin{verbatim}
var elem = document.getElementById( "foo" );

// preservation of identity
assert(jQuery(id(elem)) == jQuery(elem).id());

// preservation of composition
assert(functor(compose(f, g))
  == jQuery.compose(functor(f), functor(g)));
\end{verbatim}
\nocaptionrule \caption{Preserving identity and composition}
\label{fig:id-and-compose}
\end{figure}

Identity is trivial but composition warrants some explanation. On the left hand side the Functor promotes the already composed \textbf{Html} morphism using jQuery's \verb|map| method for operation on \verb|this|. On the right hand side each individual morphism is promoted and then \verb|jQuery.compose| performs the service of establishing \verb|this| for each execution. Most importantly the right hand side, as in Figure \ref{fig:sample-jqry-compose}, is functionally equivalent to the chaining of both methods promoted by the Functor. Given that the second Functor law holds we can say with complete confidence that two morphisms in \begin{math}hom(\mathbf{Html})\end{math} composed and then promoted into \begin{math}hom(\mathbf{Jqry})\end{math} are equivalent to the composition or chaining of two morphism in \begin{math}hom(\mathbf{Jqry})\end{math}. As a performance optimization the choice of the former is generally referred to as loop fusion or deforestation where the intermediate data structure is the mutated jQuery object set \cite{bib:deforestation}.

\section{Guidelines to Facilitate Optimization}

Given that the loop fusion optimization requires the composition of at least two morphisms from \textbf{Html} and that functionality of those morphisms is always provided to the end user as the lifted version that exists in \textbf{Jqry}, we propose the following guidelines.

\begin{enumerate}
  \item \label{item:standard-1} All methods defined on the jQuery object prototype \verb|jQuery.fn| that leverage \verb|jQuery.map| to lift an \textbf{Html} morphism into \textbf{Jqry} must provide the underlying morphism as a property for end users. Since the behavior of a given jQuery method is often too complex for the end user to determine if it meets the criteria for loop fusion, this rightly places the onus on the developers of the jQuery methods to determine and provide the necessary JavaScript function to the end user for optimization in their applications.
  \item \label{item:standard-2} All properties defined for this purpose should exist on the jQuery method itself to avoid confusion among users wishing to optimize their method chains. We propose \verb|composable| as the property name. This guarantees that jQuery methods remain the discreet units of functionality extension that they are today, and the proposed property name is semantically useful.
  \item \label{item:standard-3}Developers should, wherever possible, document and test both the \textbf{Html} morphism and its \textbf{Jqry} incarnation as discrete pieces of functionality to ensure that each \textbf{Html} morphism works independent of its \textbf{Jqry} counterpart. This ensures that the individual properties of both the \textbf{Jqry} and \textbf{Html} morphisms will persist across revisions to both.
\end{enumerate}

To assist jQuery method developers in effecting the above we also propose a small helper function that properly applies \verb|map| to an \textbf{Html} morphism, properly forwards arguments in addition to the initial \textbf{HTMLElement} argument, and sets the \verb|composable| attribute. See Appendix C.

\section{Library Aided Optimization} \label{sec:library-aided-optimization}

The reader will note that the guidelines do not attempt to differentiate the optimizable jQuery methods in any appreciable fashion other than the the possible existence of the \verb|composable| property. This is a deliberate omission in preference to an automatic identification of chains with two or more methods that define the \verb|composable| property. In fact it is possible to automatically fuse the underlying JavaScript functions but this incurs a small additional cognitive overhead and an as yet unresolved performance degradation (See Appendix D). Instead we propose a small library that will log a warning any time two or more methods are invoked in sequence when each provides the \verb|composable| property. Additionally, it will log a warning when two or more of these methods occur in a method chain but are not adjacent. While more detail on one possible implementation is provided in Appendix E, a short explanation here may aid interested parties in creating their own implementation.

Newly instantiated jQuery objects derive the bulk of their functionality from the \verb|jQuery.fn| object defined as their prototype. \verb|jQuery.fn| is also the object onto which new jQuery methods, or \textbf{Jqry} morphisms, are defined. Consequently it's possible to create a proxy object that can be inserted between a jQuery instance and \verb|jQuery.fn| in the prototype chain at runtime to record the sequence of method invocations and report opportunities for optimization.

\begin{figure}[!ht]
\begin{equation} \label{eq:jquery-default}
 jQuery \xrightharpoonup{\ \ f\ \ } jQuery.fn
\end{equation}
\begin{equation} \label{eq:jquery-with-proxy}
 jQuery \xrightharpoonup{\ \ f\ \ } Proxy \xmapsto{\ \ f\ \ } jQuery.fn
\end{equation}
\end{figure}

Taking \begin{math}\xrightharpoonup{f}\end{math} to represent the automatic prototype look-up of \begin{math}f\end{math} on the target, and \begin{math}\xmapsto{f}\end{math} to represent a invocation of \begin{math}f\end{math} on the target by the source object we have diagram \ref{eq:jquery-default} as the default jQuery behavior and diagram \ref{eq:jquery-with-proxy} as the desired behavior.

The \begin{math}Proxy\end{math} object must define it's own version of each and every function property of the \begin{math}jQuery.fn\end{math} prototype object. This allows it to count invocations of those functions and for any count greater than one raise a warning. It also allows it to invoke the method of the same name on the \begin{math}jQuery.fn\end{math} when no count is recorded, IE \begin{math}\xmapsto{f}\end{math}. Additionally the size of the jQuery object-set can be taken into account as part of configuration, as small sets of objects won't see the same benefits from composition.

The primary advantage of this approach is that it is entirely unobtrusive and requires nothing more than the inclusion of the library in an HTML document following the inclusion of jQuery itself. In this way it encourages developer adoption through ease of use.

\section{Conclusion}

Here we have clearly defined a common idiom in JavaScript using the jQuery library that can be targeted for performance optimization with a minimum of effort by developers. In addition we have established a small set of guidelines that jQuery extenders and plug-in authors can use to assist the consumers of their software in performing this optimization and provided the framework for an unobtrusive library that can automatically identify areas of potential performance degradation. In future work we hope to pursue the automatic optimization of jQuery method chains using lazy semantics to further reduce developer involvement while continuing to realize the advantages of loop fusion.

\appendix
\section{Appendix A: Each-form}

Another equally popular form of jQuery method construction leverages the jQuery built-in \verb|each| method. Converting the example from Figure \ref{fig:map-form} yields Figure \ref{fig:each-form}. The key difference being the expectation that a side effect will result from the closure that somehow leverages the information of the index and/or the \textbf{HTMLElement}.

\begin{figure}[!ht]
\small
\begin{verbatim}
jQuery.fn.eachForm = function(){
  return this.each(function( index, htmlElement ) {
    // some side effectful computation
  });
};
\end{verbatim}
\nocaptionrule \caption{General each-form}
\label{fig:each-form}
\end{figure}

\section{Appendix B: Composition of Html Morphisms}

Examples of what composition of \textbf{Html} morphisms will look like can server to reassure the reader that it behaves as necessary. In Figure \ref{fig:html-compose} an anchor element has its \verb|foo| and \verb|baz| attributes set by the newly composed \textbf{Html} morphism.

\begin{figure}[!ht]
\small
\begin{verbatim}
function a( elem ){
  elem.setAttribute( "foo", "bar" );
  return elem;
}

function b( elem ){
  elem.setAttribute( "baz", "bak" );
  return elem;
}

var elem = document.getElementById( "example-anchor" );
elem.getAttribute( "foo" ); // undefined
elem.getAttribute( "baz" ); // undefined

elem = compose( a, b )( elem );
elem.getAttribute( "foo" ); // "bar"
elem.getAttribute( "baz" ); // "bak"
\end{verbatim}
\nocaptionrule \caption{Preserving identity and composition}
\label{fig:html-compose}
\end{figure}

\section{Appendix C: Mapable Helper Function}

The \verb|jQuery.mapable| helper described in Figure \ref{fig:mapable} provides a function that can be assigned to a property on the \verb|jQuery.fn| object. It also ``tags'' the function by setting its \verb|composable| attribute to the original \textbf{Html} morphism thereby alerting developers and any libraries wishing to track optimizable \textbf{Jqry} morphisms. Additionally it does the work of forwarding any and all arguments as additional parameters to the \textbf{Html} morphism.

\begin{figure}[!ht]
\small
\begin{verbatim}
jQuery.mapable = function( htmlMorphism ){
  var jqryMorphism = function(){
    var args = arguments;

    jQuery.map(this, function( elem ){
      var newArgs = Array.prototype.slice( args );
      newArgs.unshift(elem);
      htmlMorphism.apply( elem, newArgs );
    });
  };

  jqryMorphism.composable = htmlMorphism;
  return jqryMorphism;
}j
\end{verbatim}
\nocaptionrule \caption{The mapable helper function}
\label{fig:mapable}
\end{figure}

\section{Appendix D}

An initial attempt was made to alter jQuery to support the deferral of method execution until being forced. The idea was to accumulate method calls that defined the \verb|composable| property (\verb|htmlMorphism| in the source) and then do the composition and execution all at once when necessary. While this is possible in JavaScript, the performance overhead of forwarding arguments for method invocations like those in Figure \ref{fig:args-juggle} down to the \textbf{Html} morphism in conjunction with the capability of modern JavaScript virtual machines to compile simple loops to machine code negated any positive effect of limiting the total iterations. You can view the results of a simple performance test at \url{http://jsperf.com/lazy-loop-fusion-vs-traditional-method-chaning/3}. You can also view the extension required to effect the lazy optimization at \url{https://github.com/johnbender/jquery-lazy-proxy/blob/master/lazy-proxy.js}. Further work and testing is required to completely rule out the possibility of this approach

\begin{figure}[!ht]
\small
\begin{verbatim}
jQuery( "div" ).foo( "bar", "baz" ).force();
\end{verbatim}
\nocaptionrule \caption{Forwarding string arguments}
\label{fig:args-juggle}
\end{figure}

\section{Appendix E}

A basic implementation of the object proxy described in Section \ref{sec:library-aided-optimization} can be found at \url{https://github.com/johnbender/jquery-lazy-proxy/blob/master/lazy-proxy.js} along with a simple counting and logging mechanism for identifying possible optimizations.

\bibliographystyle{abbrvnat}

% The bibliography should be embedded for final submission.

\begin{thebibliography}{}
\softraggedright

\bibitem{bib:usage}
  BuiltWith.com,
  jQuery Usage Statistics,
  \url{http://blog.builtwith.com/2011/10/31/jquery-version-and-usage-report/}
\bibitem{bib:chaining}
  jQuery.com,
  Plugin Authoring,
  Maintaining Chainability,
  \url{http://docs.jquery.com/Plugins/Authoring#Maintaining_Chainability}
\bibitem{bib:category-definition}
  Benjamin C. Pierce,
  \emph{Basic Category Theory for Computer Scientists}.
  MIT Press, Massachusets,
  First Edition,
  1991.
\bibitem{bib:htmlelement}
  W3.org,
  HTMLElement interface specification,
  \url{http://dev.w3.org/html5/spec/elements.html#htmlelement}
\bibitem{bib:all-htmlelements}
  W3.org,
  HTMLElement list,
  \url{http://dev.w3.org/html5/markup/elements.html#html-elements}
\bibitem{bib:deforestation}
  P Wadler,
  \emph{Deforestation: Transforming programs to eliminate trees}.
  Theoretical computer science,
  Elsevier,
  1990

\end{thebibliography}

\end{document}
