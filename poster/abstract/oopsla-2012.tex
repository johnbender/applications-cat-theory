%-----------------------------------------------------------------------------
%
%               Template for sigplanconf LaTeX Class
%
% Name:         sigplanconf-template.tex
%
% Purpose:      A template for sigplanconf.cls, which is a LaTeX 2e class
%               file for SIGPLAN conference proceedings.
%
% Guide:        Refer to "Author's Guide to the ACM SIGPLAN Class,"
%               sigplanconf-guide.pdf
%
% Author:       Paul C. Anagnostopoulos
%               Windfall Software
%               978 371-2316
%               paul@windfall.com
%
% Created:      15 February 2005
%
%-----------------------------------------------------------------------------


\documentclass[preprint]{sigplanconf}
\nocaptionrule

% The following \documentclass options may be useful:
%
% 10pt          To set in 10-point type instead of 9-point.
% 11pt          To set in 11-point type instead of 9-point.
% authoryear    To obtain author/year citation style instead of numeric.

\usepackage[table]{xcolor}
\usepackage[pdfborderstyle=0 0 0]{hyperref}
\usepackage{amsmath}
\usepackage{mathtools}
\usepackage[all]{xy}
\usepackage{url}

\begin{document}

\conferenceinfo{OOPSLA 2012}{October 19--26, Tucson, AZ}
\copyrightyear{2012}
\copyrightdata{[to be supplied]}

% These are ignored unless
% 'preprint' option specified.
\preprintfooter{Identifying jQuery Performance Optimizations}

\title{Automatically Identifying Performance Optimizations for jQuery Based JavaScript Applications}

\authorinfo{John Bender}
           {john.m.bender@gmail.com}
           {April, 2012}

\maketitle

\begin{abstract}
The popular \cite{bib:usage} jQuery JavaScript library makes the manipulation of HTML documents easy and intuitive, but an unfortunate side effect of its fluent interface is that an unwary programmer can easily introduce unnecessary performance overhead. And while the JavaScript execution in desktop browsers has become fast enough to hide much of the problem, the growing complexity of HTML documents and the ubiquity of web enabled mobile devices continue to make performance an important concern when developing JavaScript applications. We address this issue using category theory to identify a set of JavaScript functions and jQuery methods that can be optimized using function composition to effect loop fusion. In addition, we propose a set of guidelines and derive a library from those guidelines to help developers quickly identify opportunities to make these optimizations. As a result the developer leveraging the technique and library should be able to spend less time focusing on performance and more time on adding features and value to his or her applications.
\end{abstract}

\category{D.2.7}{Distribution, Maintenance, and Enhancement}{Enhancement}
\category{D.2.2}{Design Tools and Techniques}{Software libraries}

\terms
Performance, Design

\keywords
JavaScript, Category Theory, Loop Fusion, Optimization

\section{Introduction}

JavaScript that leverages the jQuery library can often be identified by its fluency. That is, users are encouraged to make alterations to jQuery objects by ``chaining'' methods and jQuery extension authors are counseled to always return the altered jQuery object to facilitate this form of serial method invocation \cite{bib:chaining}.

\begin{figure}[!ht]
\small
\begin{verbatim}
jQuery( "div" ).hide().addClass( "foo" ).show();
\end{verbatim}
\nocaptionrule \caption{Sample jQuery method chain}
\label{fig:jquery-sample}
\end{figure}

In Figure \ref{fig:jquery-sample}, all \textbf{HTMLDivElement}s in the document are retrieved using the \verb|"div"| CSS selector and used to instantiate a jQuery ``object-set''. They are then hidden, altered to add an additional CSS class, and shown again. Each method invocation, \verb|hide|, \verb|addClass|, and \verb|show| alters \textit{all} the elements in the jQuery object-set and then provides them for the next method to do the same. More concretely, if \begin{math}n\end{math} methods of this form are invoked in sequence it will require \begin{math}n\end{math} full iterations over the object-set.

Given that each of the methods that behave in this fashion iterates over the entirety of the jQuery object-set it's easy to see where a long sequence of them presents an opportunity for optimization by reducing the number of total iterations. Here, we hope to aid developers in reducing the number of full iterations undertaken with long method chains while otherwise maintaining the appealing fluent interface:

\begin{enumerate}
\item We define two categories \textbf{Html} and \textbf{Jqry} along with a Functor that maps from \textbf{Html} to \textbf{Jqry}. In doing so we will identify one set of JavaScript functions, the morphisms of \textbf{Html}, and their jQuery method counterparts, the morphisms of \textbf{Jqry}, that can be composed while remaining confident in the effects of their execution.
\item We define a simple standard for jQuery method and plug-in developers to follow that will allow end users to manually optimize their method chains where appropriate.
\item We construct a simple and unobtrusive utility for JavaScript developers using jQuery that will alert them to possible opportunities for optimization when chaining methods that adhere to the proposed guidelines.
\end{enumerate}

\section{Html, Jqry, and a Functor}

The goal of introducing category theory is to establish a more rigorous definition of the set of JavaScript functions that meet the criteria for optimization using loop fusion. This implementation follows the well documented pattern established in other programming language environments like Haskell [!!].

\textbf{Html} is the category of objects as HTMLElements provided by any standards compliant JavaScript DOM API \cite{bib:htmlelement, bib:all-htmlelements} and morphisms as JavaScript functions from HTMLElements to HTMLElements. Meanwhile, \textbf{Jqry} is the category of jQuery objects wrapping HTMLElements and morphisms as JavaScript methods defined on jQuery objects that return jQuery objects. Each can easily be shown to satisfy the category requirements of identity, composition, and closure under composition.

To complement the categories we define a mapping between them, from \textbf{Html} to \textbf{Jqry}, which delivers on the promise of a concrete set of JavaScript functions necessary to proceed with optimizing jQuery method chains. Specifically, if it can be shown that composition is preserved when morphisms from \textbf{Html} are translated into \textbf{Jqry} using jQuery's \verb|map| it can also be shown that the chained \textbf{Jqry} equivelants can be replaced by a single composed \textbf{Html} morphism translated into \textbf{Jqry} (Figure \ref{fig:compose}).

\begin{figure}[!ht]
\small
\begin{verbatim}
var foo = jQuery( "#foo" );

assert(foo.F().G() == jQuery.map(foo, cmps(f, g)));
\end{verbatim}
\nocaptionrule \caption{Preserving composition}
\label{fig:compose}
\end{figure}


\section{A Standard to Facilitate Optimization}

Given that the loop fusion optimization requires the composition of at least two morphisms from \textbf{Html} and that functionality of those morphisms is always provided to the end user as the translated version that exists in \textbf{Jqry}, we propose the following guidelines.

\begin{enumerate}
  \item \label{item:standard-1} All methods defined on the jQuery object prototype \verb|jQuery.fn| should leverage \verb|jQuery.map| to lift an \textbf{Html} morphism into \textbf{Jqry}. In addition to setting the groundwork for providing the underlying \textbf{Html} morphism this enforces consistency and readability so that when the morphism isn't provided seperately end users can attempt to provide their own.
  \item \label{item:standard-2} All methods defined for this purpose should expose the underlying \textbf{Html} morphism as the \verb|composable| property on the jQuery method itself. This function should return a function with it's arguments partially applied that will expect only an HTMLElement. Providing a consistent property makes fusion easier for the end user and enables automatic tooling like the sample identification library detailed below.
  \item \label{item:standard-3}Developers should, wherever possible, document and test the performance of both the \textbf{Html} morphism and its \textbf{Jqry} incarnation. This ensures that the work performed in the name of performance will indeed yield the benefits that the end user will expect.
\end{enumerate}

To assist jQuery method developers in effecting the above we also propose a small helper function that properly applies \verb|map| to an \textbf{Html} morphism, properly forwards arguments in addition to the initial \textbf{HTMLElement} argument, and sets the \verb|composable| attribute. See Appendix C.

\section{Library Aided Optimization} \label{sec:library-aided-optimization}

The reader will note that the guidelines do not attempt to differentiate the optimizable jQuery methods in any appreciable fashion other than the the possible existence of the \verb|composable| property. This is a deliberate omission in preference to an automatic identification of chains with two or more methods that define the \verb|composable| property. In fact it is possible to automatically fuse the underlying JavaScript functions but this incurs a small additional cognitive overhead and an as yet unresolved performance degradation (See Appendix D).\\

[\textbf{TODO}] further discuss automatic fusion and possible benefits to end users.

[\textbf{TODO}] explore the reasons for the performance degredation with the automatic approach as it may be relevant to the hand fused approach.\\

Instead we propose a small library that will log a warning any time two or more methods are invoked in sequence when each provides the \verb|composable| property. Additionally, it will log a warning when two or more of these methods occur in a method chain but are not adjacent. While more detail on one possible implementation is provided in Appendix E, a short explanation here may aid interested parties in creating their own implementation.

Newly instantiated jQuery objects derive the bulk of their functionality from the \verb|jQuery.fn| object defined as their prototype. \verb|jQuery.fn| is also the object onto which new jQuery methods, or \textbf{Jqry} morphisms, are defined. Consequently it's possible to create a proxy object that can be inserted between a jQuery instance and \verb|jQuery.fn| in the prototype chain at runtime to record the sequence of method invocations and report opportunities for optimization.

\begin{figure}[!ht]
\begin{equation} \label{eq:jquery-default}
 jQuery \xrightharpoonup{\ \ f\ \ } jQuery.fn
\end{equation}
\begin{equation} \label{eq:jquery-with-proxy}
 jQuery \xrightharpoonup{\ \ f\ \ } Proxy \xmapsto{\ \ f\ \ } jQuery.fn
\end{equation}
\end{figure}

[\textbf{TODO}] add diagram to illustrate prototype chain alteration
\\

Taking \begin{math}\xrightharpoonup{f}\end{math} to represent the automatic prototype look-up of \begin{math}f\end{math} on the target, and \begin{math}\xmapsto{f}\end{math} to represent a invocation of \begin{math}f\end{math} on the target by the source object we have diagram \ref{eq:jquery-default} as the default jQuery behavior and diagram \ref{eq:jquery-with-proxy} as the desired behavior.

The \begin{math}Proxy\end{math} object must define it's own version of each and every function property of the \begin{math}jQuery.fn\end{math} prototype object. This allows it to count invocations of those functions and for any count greater than one raise a warning. It also allows it to invoke the method of the same name on the \begin{math}jQuery.fn\end{math} when no count is recorded, IE \begin{math}\xmapsto{f}\end{math}. Additionally the size of the jQuery object-set can be taken into account as part of configuration, as small sets of objects won't see the same benefits from composition.

The primary advantage of this approach is that it is entirely unobtrusive and requires nothing more than the inclusion of the library in an HTML document following the inclusion of jQuery itself. In this way it encourages developer adoption through ease of use.

\section{Initial Experimental Results}

[\textbf{TODO}] simple performance examples illustrating fusion benefits. \href{http://jsperf.com/lazy-loop-fusion-vs-traditional-method-chaning/5}{Sample}

\section{Conclusion}

Here we have clearly defined a common idiom in JavaScript using the jQuery library that can be targeted for performance optimization with a minimum of effort by developers. In addition we have established a small set of guidelines that jQuery extenders and plug-in authors can use to assist the consumers of their software in performing this optimization and provided the framework for an unobtrusive library that can automatically identify areas of potential performance degradation. In future work we hope to pursue the automatic optimization of jQuery method chains using lazy semantics to further reduce developer involvement while continuing to realize the advantages of loop fusion.

\bibliographystyle{abbrvnat}

% The bibliography should be embedded for final submission.

\begin{thebibliography}{}
\softraggedright

\bibitem{bib:usage}
  BuiltWith.com,
  jQuery Usage Statistics,
  \url{http://blog.builtwith.com/2011/10/31/jquery-version-and-usage-report/}
\bibitem{bib:chaining}
  jQuery.com,
  Plugin Authoring,
  Maintaining Chainability,
  \url{http://docs.jquery.com/Plugins/Authoring#Maintaining_Chainability}
\bibitem{bib:category-definition}
  Benjamin C. Pierce,
  \emph{Basic Category Theory for Computer Scientists}.
  MIT Press, Massachusets,
  First Edition,
  1991.
\bibitem{bib:htmlelement}
  W3.org,
  HTMLElement interface specification,
  \url{http://dev.w3.org/html5/spec/elements.html#htmlelement}
\bibitem{bib:all-htmlelements}
  W3.org,
  HTMLElement list,
  \url{http://dev.w3.org/html5/markup/elements.html#html-elements}
\bibitem{bib:deforestation}
  P Wadler,
  \emph{Deforestation: Transforming programs to eliminate trees}.
  Theoretical computer science,
  Elsevier,
  1990

\end{thebibliography}

\end{document}
