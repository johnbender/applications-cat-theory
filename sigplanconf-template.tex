%-----------------------------------------------------------------------------
%
%               Template for sigplanconf LaTeX Class
%
% Name:         sigplanconf-template.tex
%
% Purpose:      A template for sigplanconf.cls, which is a LaTeX 2e class
%               file for SIGPLAN conference proceedings.
%
% Guide:        Refer to "Author's Guide to the ACM SIGPLAN Class,"
%               sigplanconf-guide.pdf
%
% Author:       Paul C. Anagnostopoulos
%               Windfall Software
%               978 371-2316
%               paul@windfall.com
%
% Created:      15 February 2005
%
%-----------------------------------------------------------------------------


\documentclass[preprint]{sigplanconf}

% The following \documentclass options may be useful:
%
% 10pt          To set in 10-point type instead of 9-point.
% 11pt          To set in 11-point type instead of 9-point.
% authoryear    To obtain author/year citation style instead of numeric.

\usepackage{amsmath}

\begin{document}

\conferenceinfo{OOPSLA 2012}{October 19--26, Tucson, AZ}
\copyrightyear{2012}
\copyrightdata{[to be supplied]}

\titlebanner{banner above paper title}        % These are ignored unless
\preprintfooter{short description of paper}   % 'preprint' option specified.

\title{Automatic Performance Optimizations for jQuery Based JavaScript Applications}

\authorinfo{John Bender}
           {}
           {john.m.bender@gmail.com}
\maketitle

\begin{abstract}
The jQuery JavaScript library makes the manipulation of the Document Object Model easy and intuitive via chained method calls, but an unfortunate side effect of its fluent interface is that an unwary programmer can easily introduce performance issues. Though desktop browsers have become fast enough to hide much of the problem, the growing complexity of HTML documents and the ubiquity of alternative environments in which JavaScript is now runnning have returned performance to a place of importance. The solution presented achieves a dramatic reduction in object-set iterations in existing software while minimizing the overhead required from the developer to realize its benefits. As a result the developer can spend less time focusing on performance and more time on adding features and value to his or her applications.
\end{abstract}

\category{D.2.7}{Distribution, Maintenance, and Enhancement}{Enhancement}
\category{D.2.2}{Design Tools and Techniques}{Software libraries}

\terms
Performance, Design

\keywords
JavaScript, Category Theory, Loop Fusion, Optimization

\section{Introduction}

The text of the paper begins here.

\appendix
\section{Appendix Title}

This is the text of the appendix, if you need one.

\acks

Acknowledgments, if needed.

% We recommend abbrvnat bibliography style.

\bibliographystyle{abbrvnat}

% The bibliography should be embedded for final submission.

\begin{thebibliography}{}
\softraggedright

\bibitem[Smith et~al.(2009)Smith, Jones]{smith02}
P. Q. Smith, and X. Y. Jones. ...reference text...

\end{thebibliography}

\end{document}
